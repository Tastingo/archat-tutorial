\documentclass[12pt]{article}
\usepackage{xeCJK}

\usepackage{charter}
\usepackage{fullpage}
\usepackage[colorlinks=false]{hyperref}
\usepackage{ifthen}
\usepackage{comment}
\usepackage[title,titletoc]{appendix}
\usepackage{pagecolor}
\usepackage{amsmath}
\usepackage{amsfonts}
%\usepackage[normalem]{ulem}
\usepackage{siunitx}
\usepackage{amsthm}
\usepackage{paralist}
\usepackage[numbers,sort&compress]{natbib}
\sisetup{per=slash, load=abbr}

\usepackage{pgfplots}
\usetikzlibrary{positioning}
\usetikzlibrary{fit}
\usetikzlibrary{snakes}
\usetikzlibrary{shapes.geometric}
\usetikzlibrary{patterns}
\usetikzlibrary{shapes,arrows,chains}
\usepgfplotslibrary{patchplots,colormaps}
\usetikzlibrary{calc}
\usetikzlibrary{positioning, fit}
\usetikzlibrary{backgrounds}
\usetikzlibrary{intersections}

\newcommand{\whitepaper}[1]{\begin{center}\fbox{\parbox{0.75\textwidth}{{\small
#1}}}\end{center}}

\newcommand{\pcolor}{white!25}

\usepackage{setspace}
\usepackage{algorithm2e}
\bibliographystyle{ieeetr}

\usepackage{geometry}
\geometry{left=3cm,right=3cm,top=1.6cm,bottom=3cm,headheight=0pt,headsep=1.5em}
\usepackage{fancyhdr}
\pagestyle{fancy}

\usepackage{indentfirst}


\setCJKmainfont[BoldFont = STSongti-SC-Bold]{STSongti-SC-Regular}
\setCJKfamilyfont{hei}{SIL-Hei-Med-Jian}		%宋体
\setCJKfamilyfont{song}{SimSun}		%宋体
\setCJKfamilyfont{kai}{Kaiti}		%楷体
\setCJKfamilyfont{fang}{song}	%仿宋
\setCJKfamilyfont{li}{song}			%隶书
\setCJKfamilyfont{you}{Yuanti}		%幼圆

\newcommand{\song}{\CJKfamily{song}}	%宋体
\newcommand{\hei}{\CJKfamily{hei}}	%黑体
\newcommand{\kai}{\CJKfamily{kai}}	%楷体
\newcommand{\fs}{\CJKfamily{fang}}	%仿宋
\newcommand{\li}{\CJKfamily{li}}		%隶书
\newcommand{\you}{\CJKfamily{you}}	%幼圆
\newcommand{\reffig}[1]{图\ref{#1}}
\newcommand{\refsec}[1]{\S \ref{#1}}

\onehalfspacing   % ----------设置1.5倍行距(可能有意义,待调整)
\setlength{\parindent}{2.1em}
\setlength{\parskip}{0.3\baselineskip}
\newcommand{\dom}{{\; \texttt{dom}\;}}

\setCJKsansfont[BoldFont = STHeitiSC-Medium]{STHeitiSC-Light}


\newtheorem{property}{特征}
%\addbibresource{reference.bib}

\begin{document}
\pagestyle{empty}
\renewcommand{\contentsname}{目录}
\renewcommand{\abstractname}{摘要}
\renewcommand{\refname}{发表文献}
%\renewcommand{\nomname}{术语表(按首字母排序)}
\renewcommand{\figurename}{图}
\renewcommand{\tablename}{表}
\renewcommand{\baselinestretch}{1.5}
\renewcommand{\appendixname}{附录}
\renewcommand{\proofname}{证明}

\pagecolor{\pcolor}


  \begin{center}
    \vspace*{0.5cm}
    \vspace{0.5cm}
    \textbf{\huge{CAT介绍}}
    \vspace{0.5cm}
    \textbf{}
  \end{center}
\setcounter{page}{0}
%\thispagestyle{empty}
%\tableofcontents
\pagestyle{fancy}
\vspace*{0.01cm}
%main text
CAT,全称Chat Token,是AR聊天室运营所需的唯一代币。
\section*{CAT的作用}

CAT是用于进入AR聊天室所需的凭证。

\section*{AR聊天室收费机制}
\begin{itemize}
\item AR>26的地址(对应以太坊排名约前10000)进入聊天室不需要要支付任何CAT。
\item AR<26的地址在一个周期内(约3天)进入聊天室需支付$6$CAT。
\end{itemize}
\section*{如何获得CAT}
\begin{enumerate}
\item 每个周期在所有AR>26的地址中,随机抽取100个地址空投$6$CAT
\item 每个周期对进入聊天室的地址给予等同于其AR的CAT
\item 通过调用特定合约的exchange方法用ETH购买
\end{enumerate}
注:用户需要先支付6CAT进入聊天室,随后才会能获得上述第二条给予的CAT

\section*{合约购买规则}
合约采用“单曲线”模型定价,即购买CAT所需的ETH与累计已售出CAT数目有关。

曲线定义:$ y=ax
$

其中$x$为已售出代币数量,单位为HU$=10^{-3}$ CAT, $y$为当前状况下(已售出$x$ HU),购买一HU需要支付的ETH数量(单位为wei$=10^{-18}$ETH)。系数$a$定义为7142800

举例:累计卖出100个CAT时,$x=100*10^3=10^5$,此时一个HU价值约
$10^5*7142800*10^{-18}=7.1428*10^{-7}$ ETH$=10^{-3} $RMB,即相当于1个CAT价值1RMB。

同时,已售出HU数量$x$随时间自动触发衰减,目前速率为每隔5个以太坊区块高度(约75s)$x$减少1。在此设定下1天$x$大约减少1000,即已售出CAT数量每天减1。

\section*{分析}
以下分析均针对一个周期而言。
 \begin{itemize}
\item AR>26(对应前10000名)的用户可免费进入聊天室,且每个周期领到最高26的CAT
\item 6<AR<26(对应10000-20000名)的用户需要支付$6$ CAT以进入聊天室,进入之后随即能领到等于其AR的CAT,因为AR>6故此类用户通过AR聊天室能获得正收益。
\item AR<6(对应20000名以后(的用户需要支付$6$ CAT门票以进入聊天室,进入之后随即领到等于其AR的CAT,因AR<6故此类用户无法获得正收益,故需要向合约或者其他人购买CAT才能持续享受AR聊天室服务。
\end{itemize}

\end{document}
